% !TEX root = ../main.tex

\chapter{绪论}
\label{cha:intro}


\section{研究背景与意义}
\label{sec:meaning}
% 挑战 --- 工业转型对工业系统安全性的要求越来越高。
"工业4.0"的概念在2013年被正式提出之后,“智能化”随即成为引领第四代工业革命的主题,各国纷纷制定政策以做好准备迎接工业转型的挑战与机遇。2011年在德国汉诺威工业博览会上“工业4.0”的概念首度被三位教授提出。2013年4月在汉诺威工博会上《保障德国制造业的未来:关于实施“工业4.0”战略建议书》发布,至此,“工业4.0”成为《德国2020高技术战略》的十大未来项目之一并正式上升为国家战略。\cite{kagermann2013recommendations}。美国随后颁布了《先进制造业合作伙伴》战略\cite{advanced2012capturing}。我国政府颁布了以实现制造强国为战略目标的《中国制造2025》政策\cite{周济2015智能制造——“中国制造}。随着工业转型,更智能化的工业系统在国家基础建设和发展进程中将扮演更重要的角色,安全问题不容忽视,《中国制造2025》中“质量为先”的基本方针更是充分体现了安全问题在工业转型中的重要性。

% 挑战 ---工业现场安全可靠的要求
工业生产的顺利进行是维护国民经济和生活保障的重要基础,保证工业生产现场和工业装备运行现场安全可靠的运行至关重要。工业生产和工业装备的安全问题,既是企业产品质量的重要标志,也是全社会关注的重点。1984年设在印度博帕尔市的美国联合炭化物公司开办的一家农药厂发生的毒气泄漏事故(死亡2500人,中毒125000人)、1986年两起震惊世界的巨大事故(美国“挑战者”号航天飞机爆炸和前苏联切尔诺贝利核电站爆炸事故)等,使人们对安全问题有了更加深入的认识\cite{工业安全评价方法与矿井安全评价技术综述}。以铁路为例,随着我国铁路提速和基础设施技术改造,列车的运行速度和密度日益增大,铁路运输生产的系统复杂度和风险度显著提高。为了适应现代生产生活发展的需要,企业安全监察部门应当尽可能的建立起安全预警系统,实现安全管理模式由“事故出发型”向“事故发现型”的转变\cite{铁路安全预警系统的研究与涉及}。保证复杂的工业系统可靠、安全、可持续的运行面临巨大的挑战。

% 机遇 ---- 监控设备的发展,深度学习技术的成熟,嵌入式设备的进步
随着社会的进步和科技的发展,视频监控也经历者从模拟化向数字化、网络化、智能化的革命,通过视频监控系统就能够快速、实时的进行监控和监视。
视频图像分析技术作为智能监控系统的核心技术,正广泛应用于公共安全监控领域。随着以摄像机为核心的视频监控系统成本日益降低,视觉监控系统在我国开始步入普及阶段。人脸识别在监控系统中发挥了重要作用因为它不需要其他物体的协助\cite{face2012recoginition}。视频和图像数据是人们体验最直观的数据类型,可以从图像中获取高层次的信息,比如通过观察列车受电弓接触点位置可以得知受电弓接触网是否接触良好后者存在脱网等安全隐患,通过观察列车驾驶员的操作动作得知操作的规范性和驾驶员的技术水平。

近年来对着深度学习技术的火热发展,图像数据的目标检测算法也从基于手工特征的传统算法转向了基于深度学习的检测算法\cite{deeplearing2018survey}。深度学习技术和图像检测算法的发展使得对工业现场的图像进行可靠高效的分析成为可能。而在硬件方面,嵌入式处理器的计算能力不断提升,已经广泛地应用于我们生活的各个领域,例如:计算机,汽车,航天飞机等。

% 挑战 --- 视频安全监控
但是在工业生产现场,视觉监控系统的功能仅仅停留在检查人员对视频信号的人工监视和事后录像分析上,并没有充分利用目前计算机技术告诉发展所提供的智能算法和强大的计算能力\cite{面向视觉监视实时跟踪的动态背景更新方法}。工业生产现场的视频监控系统迫切需要对危险状态进行线实时的分析,以实现对生产运行环境的可知可控。从视频监控系统采集的图像中利用深度学习技术学习检测风险是实现生产环境隐患实时监测的核心环节,然而工业图像普遍存在像素低,信噪比低的特点,导致标注难和分析难的问题;再者工业系统运行时故障属于小概率事件,鲁棒的图像检测算法需要不断的对新产生的数据进行学习分析,需要设计能够对不断产生的监控数据在线学习的算法,才能保证日后的检测任务中保证有效性。工业生产的危险预警重点在于及时预测,基于深度学习的模型通常需要大量的计算和内存访问,嵌入式处理器的计算能力相对较差,需要对深度学习模型进行压缩优化,得到最佳的网络结构,最终可以在嵌入式设备中实时运行。
如果能利用视频监控系统监控工业生产现场的状态,并对监控的视频数据进行实时分析,那么工业生产现场及设备一旦出现危险状况,可以及时预警处理,避免重大的生命财产损失。

综上所述,本文主要研究工业图像检测算法,并将检测模型进行压缩优化,最终在工业现场的嵌入式设备达到实时检测的要求,同时利用现场不断采集的新数据进行模型的自我学习更新,以适应工业现场复杂多变的特点。因此本文的研究对于基于图像数据及时检测工业生产现场的危险状态,发现并预测潜在故障隐患,建立依靠先进技术支撑的故障预警系统,具有重大意义。

\section{研究内容与贡献}
\label{sec:content-contribution}

%突出我的方法相对于传统方法的先进性。理论贡献,公开benchmark,在实验室的列车故障数据上进行验证,具有现实意义
\subsection{研究内容}
本文针对复杂多变的工业图像的检测算法及其加速技术展开系统研究。首先论述了研究的北京和意义,然后然后梳理并综述相关研究工作。接下来针对3个现实问题依次展开工业图像检测算法研究与应用的工作:(1)针对复杂的工业现场图像数据,以及现有的图像检测算法只对单张图像独立分析的问题,提出预测式的图像检测算法,结合前序图像的状态的估计值和当前图像的预测值,对该时刻的图像状态进行估计,并在列车受电弓数据上展开对比实验;(2)针对工业现场的监控数据不断更新,情况复杂多变的挑战,以及基于深度学习的模型训练周期长,需要定期更新的问题,提出基于当前模型恒等变换的在线训练方法,该方法快速的将小模型的知识迁移到新的模型,并且不断对新的数据进行学习。(3)而基于深度学习的图像检测模型参数量和计算量通常较大,在嵌入式设备中无法实时运行的问题,提出通过对卷积神经网络进行剪枝来搜索最佳的网络结构,极大的加快了检测效率。最后进行总结与展望。

论文的主要研究内容和结构安排如下:

第 1 章:绪论
	
首先论述研究背景和意义;其次介绍本文主要研究内容和贡献。

第 2 章:相关研究综述

对相关内容研究,深度学习、图像检测算法,训练加速算法、卷积神经网络剪枝、嵌入式系统、以及模型移植等,进行梳理和综述;论述现有工作的优缺点以及与本文工作的契合点。

第 3 章:基于深度学习的预测式工业图像检测算法

针对工业视频和图像场景复杂的挑战,以及现有图像检测算法无法利用前序图像信息的问题,本文利用以列车受电弓图像为代表的工业图像数据进行研究实验。首先介绍研究背景与数据分析工作;然后提出基于深度学习方法和前序图像的预测式图像检测算法,并进行对比实验分析。

第 4 章:基于模型恒等变换的在线训练算法

针对工业现场的监控数据不断更新,情况复杂多变的挑战,以及现有的深度学习模型训练方法大多是利用采集好的数据从零开始训练,模型更新周期长的问题。本文提出基于模型恒等变换的在线训练方法,首先对当前模型结构加宽加深得到新的模型结构,同时不改变模型的映射关系,利用不断采集的新数据对新的模型进行微调,最后在2个公开的图像数据集上进行对比实验分析。

第 5 章:基于剪枝搜索的深度模型嵌入式移植优化算法

针对而基于深度学习的图像检测模型参数量和计算量较大的挑战,目前难以在嵌入式设备中实时运行。本文针对工业现场常用的ARM平台,首先进行卷积核的剪枝探索最优的网络结构,接下来逐层测试卷积神经网络前向传递过程的计算时间和内存访问时间,最终得到最佳的网络结构。

第 6 章:总结与展望

对本文工作以及相关结论进行概要总结;对未来工作进行展望。

\subsection{研究贡献}

1.本文基于以列车受电弓监控图像为代表的工业图像数据,提出基于深度学习方法和前序图像的预测式图像检测算法,进一步利用前序图像的状态信息对当前时刻的图像进行检测,在一定程度上弥补了现有连续图像检测算法的不足。

2.本文提出基于模型恒等变换的在线训练方法,该方法快速的将小模型的知识迁移到新的模型,缩短新模型的训练时间。本文在2个的图像数据集展开实验,证明该方法的有效性。

3.本文提出基于剪枝搜索的深度模型嵌入式移植优化算法,对卷积核剪枝以探索最优的网络结构,并在ARM平台台逐层测试卷积神经网络前向传递过程的计算时间和内存访问时间,最终得到最佳的网络结构。

4.本文所有方法均得到了真实场景数据或公开数据的验证,具有较好的应用价值和借鉴意义。


