\begin{resume}

  \resumeitem{个人简历}

  1992 年 4 月 27 日出生于 广西壮族自治区横县。

  2011 年 9 月考入 广西 大学 计算机与电子信息学院 信息安全专业,2015 年 7 月本科毕业并获得工学学士学位。

  2015 年 9 月免试进入清华大学软件学院攻读硕士学位至今。

  %\researchitem{发表的学术论文} % 发表的和录用的合在一起

  % 1. 已经刊载的学术论文(本人是第一作者,或者导师为第一作者本人是第二作者)
  % \begin{publications}
  %   \item Yang Y, Ren T L, Zhang L T, et al. Miniature microphone with silicon-
  %     based ferroelectric thin films. Integrated Ferroelectrics, 2003,
  %     52:229-235. (SCI 收录, 检索号:758FZ.)
  % \end{publications}

  % % 2. 尚未刊载,但已经接到正式录用函的学术论文(本人为第一作者,或者
  % %    导师为第一作者本人是第二作者)。
  % \begin{publications}[before=\publicationskip,after=\publicationskip]
  %   \item Yang Y, Ren T L, Zhu Y P, et al. PMUTs for handwriting recognition. In
  %     press. (已被 Integrated Ferroelectrics 录用. SCI 源刊.)
  % \end{publications}

  % % 3. 其他学术论文。可列出除上述两种情况以外的其他学术论文,但必须是
  % %    已经刊载或者收到正式录用函的论文。
  % \begin{publications}
  %   \item Wu X M, Yang Y, Cai J, et al. Measurements of ferroelectric MEMS
  %     microphones. Integrated Ferroelectrics, 2005, 69:417-429. (SCI 收录, 检索号
  %     :896KM)
  % \end{publications}

  %\researchitem{研究成果} % 有就写,没有就删除
  % \begin{achievements}
  %   \item 任天令, 杨轶, 朱一平, 等. 硅基铁电微声学传感器畴极化区域控制和电极连接的
  %     方法: 中国, CN1602118A. (中国专利公开号)
  % \end{achievements}

\end{resume}
