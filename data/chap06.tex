% !TEX root = ../main.tex

\chapter{总结与展望}
\label{chap:conclusion}

本文针对“高维、高噪、时变”时间序列数据的特征学习以及其在工业装备运维中的作用进行了系统研究。主要研究工作如下:

1. 提出了基于符号回归的系统结构特征学习方法并展开应用。利用以轴温为代表的设备正常态运行数据展开研究;学习了系统结构特征,揭示了轴承的动态性和信号间的相互作用关系;将最优系统结构特征作为系统运行时的健康基线,设计了在线实时异常检测框架。

2. 提出了基于复杂系统临界相变理论的早期预警特征学习方法并展开应用。利用来自4个不同工业系统的系统全生命周期数据开展研究;学习了具有普适性的系统失效早期预警特征;针对早期特征,提出了关键跳变检测算法实现故障预测。

3. 提出了基于生成式对抗网络的无监督式时间序列特征学习模型——TSGAN,并展开应用。利用最大的时间序列分类数据集(85个数据集)开展研究;一方面,TSGAN 的生成器学习了真实数据集的分布,作为模拟器使用,生成了逼真的多样化的时间序列;另一方面,将TSGAN的判别器作为特征转换器与简单分类器结合构建了半监督式时间序列分类框架,并取得优秀效果。

未来工作展望:

1. 对基于符号回归的系统结构特征学习框架进一步优化。一方面,当前的方法存在产生冗余运算子的情况,仍有进一步优化的空间。另一方面,尝试融合更多的学习策略或模型进一步提高现有框架的数据和环境自适应能力。

2. 对复杂系统临界相变预测理论的更多探索。一方面,基于现有发现研究通用方法来预测和预防系统失效很有意义。另一方面,探索更多的方法,尤其是先进的机器学习方法,发现更多临界相变普适特征很有吸引力。

3. 对TSGAN的进一步改进和应用。一方面,通用配置的TSGAN对少量的极端验证集(比如高噪振动信号)仍不够鲁棒,因此可以尝试更多的神经网络结构或优化方法对TSGAN进行改进;另一方面,TSGAN作为特征转换器,不仅可以支持时间序列分类,还可以尝试与更多的任务结合,比如聚类、切割、索引、异常检测、主题发现等任务。

4. 临界相变理论和生成式模型的结合。一方面,临界相变理论揭示了系统底层失效机理,若看似不同的系统可以临界相变理论为媒介实现互联互通,将具有领域突破性意义;另一方面,生成式模型具有学习数据底层分布的能力,可对无标记数据进行利用,若能通过生成式模型揭示数据底层共享特征域并基于此实现不同数据间的转换,将有可能大幅度缓解标记数据缺乏问题。探索以上任一个方面或综两方面的研究都极具吸引力。

%5. 时间序列特征学习方法,尤其是无监督式时间序列特征学习方法的深入。特征学习是机器学习和数据挖掘的关键,本文的研究只涉及
