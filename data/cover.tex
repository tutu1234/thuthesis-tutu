\thusetup{
  %******************************
  % 注意:
  %   1. 配置里面不要出现空行
  %   2. 不需要的配置信息可以删除
  %******************************
  %
  %=====
  % 秘级
  %=====
  secretlevel={秘密},
  secretyear={10},
  %
  %=========
  % 中文信息
  %=========
  ctitle={工业图像的检测算法及其加速\\技术研究与应用},
  cdegree={工程硕士},
  cdepartment={软件学院},
  cmajor={软件工程},
  cauthor={胡志坤},
  csupervisor={邓仰东副研究员},
  %cassosupervisor={陈文光教授}, % 副指导老师
  %ccosupervisor={某某某教授}, % 联合指导老师
  % 日期自动使用当前时间,若需指定按如下方式修改:
   cdate={二〇一九年五月},
  %
  % 博士后专有部分
  cfirstdiscipline={计算机科学与技术},
  cseconddiscipline={系统结构},
  postdoctordate={2009年7月——2011年7月},
  id={编号}, % 可以留空: id={},
  udc={UDC}, % 可以留空
  catalognumber={分类号}, % 可以留空
  %
  %=========
  % 英文信息
  %=========
  etitle={Research and Application on Industrial Image Detection Algorithm and Its Acceleration Technology},
  % 这块比较复杂,需要分情况讨论:
  % 1. 学术型硕士
  %    edegree:必须为Master of Arts或Master of Science(注意大小写)
  %             “哲学、文学、历史学、法学、教育学、艺术学门类,公共管理学科
  %              填写Master of Arts,其它填写Master of Science”
  %    emajor:“获得一级学科授权的学科填写一级学科名称,其它填写二级学科名称”
  % 2. 专业型硕士
  %    edegree:“填写专业学位英文名称全称”
  %    emajor:“工程硕士填写工程领域,其它专业学位不填写此项”
  % 3. 学术型博士
  %    edegree:Doctor of Philosophy(注意大小写)
  %    emajor:“获得一级学科授权的学科填写一级学科名称,其它填写二级学科名称”
  % 4. 专业型博士
  %    edegree:“填写专业学位英文名称全称”
  %    emajor:不填写此项
  edegree={Master of Engineering},
  emajor={Software Engineering},
  eauthor={Hu Zhikun},
  esupervisor={Associate Research Fellow Deng Yangdong},
  %eassosupervisor={Chen Wenguang},
  % 日期自动生成,若需指定按如下方式修改:
  edate={May, 2019},
  %
  % 关键词用“英文逗号”分割
  ckeywords={工业图像,图像检测,在线训练,模型加速,嵌入式应用},
  ekeywords={Industrial Image, Image Detection, training online, model acceleration, Embedded application}
}
% 定义中英文摘要和关键字
\begin{cabstract}
    工业生产现场封闭复杂,工业装备通常运行在高温高压的环境中。随着社会的进步和科技的发展,在工业生产现场,通过视频监控系统就能够快速、实时的进行监控和监视,避免人工操作的安全风险,也能及时监控工业现场状态。本文主要针对工业现场的图像检测算法以及其在工业现场的嵌入式设备中在线训练和实时运行的加速技术进行研究,最终能够利用工业现场的嵌入式设备对现场状态进行实时监控。首先基于深度学习的方法对铁路列车受电弓与供电线路的接触点位置监控视频进行预测式监控,其次设计检测模型在线训练算法实现检测模型利用工业现场不断产生新数据进行更新优化,最后为保证算法在嵌入式设备中实时运行,对模型结构进行压缩加速。
  
  现有的图像检测算法主要是基于深度学习的方法对单张图像进行检测,而工业现场采集的数据主要以连续的视频形式。本文针对针对工业现场图像数据提出预测式的图像检测算法,相比单独的图像检测算法,结合前序图像的状态的估计值和当前图像的预测值,对该时刻的图像状态进行估计,本文在受电弓接触点数据集上进行实验,以单张图像检测模型作为基线,所提出的预测式的检测算法在精度提升8\%。
  
  工业现场的监控数据不断更新,情况复杂多变。现有的深度学习模型训练方法大多是利用采集好的数据从零开始训练,模型更新周期长,需要定期部署。本文提出基于当前模型恒等变换的在线训练方法,首先对当前模型结构加宽加深得到新的模型结构,同时不改变模型的映射关系,接下来利用不断采集的新数据对新的模型进行微调,该方法快速的将小模型的知识迁移到新的模型,并且不断对新的数据进行学习。本文在2个的图像数据集展开实验,证明该方法的有效性,在不影响准确率的前提下,所提出的在线训练算法将训练时间至少缩短20\%。
  
  工业现场的监控要求具有实时性,如受电弓接触点如果出现故障需要立刻报警处理,而基于深度学习的图像检测模型参数量和计算量通常较大,在工业现场的嵌入式设备中无法满足实时运行的要求。本文针对工业现场常用的ARM平台,首先进行卷积核的剪枝探索最优的网络结构,接下来逐层测试卷积神经网络前向传递过程的计算时间和内存时间,极大的加快了检测效率,在ARMv7的实验平台上对受电弓接触点的检测达到实时效果。
  
\end{cabstract}

% 如果习惯关键字跟在摘要文字后面,可以用直接命令来设置,如下:
% \ckeywords{\TeX, \LaTeX, CJK, 模板, 论文}

\begin{eabstract}

English abstract

\end{eabstract}

% \ekeywords{\TeX, \LaTeX, CJK, template, thesis}
