\thusetup{
  %******************************
  % 注意:
  %   1. 配置里面不要出现空行
  %   2. 不需要的配置信息可以删除
  %******************************
  %
  %=====
  % 秘级
  %=====
  secretlevel={秘密},
  secretyear={10},
  %
  %=========
  % 中文信息
  %=========
  ctitle={工业时间序列数据的特征学习\ \ 技术研究与应用},
  cdegree={工程硕士},
  cdepartment={软件学院},
  cmajor={软件工程},
  cauthor={黄凡玲},
  csupervisor={邓仰东副研究员},
  %cassosupervisor={陈文光教授}, % 副指导老师
  %ccosupervisor={某某某教授}, % 联合指导老师
  % 日期自动使用当前时间,若需指定按如下方式修改:
  cdate={二〇一八年五月},
  %
  % 博士后专有部分
  cfirstdiscipline={计算机科学与技术},
  cseconddiscipline={系统结构},
  postdoctordate={2009年7月——2011年7月},
  id={编号}, % 可以留空: id={},
  udc={UDC}, % 可以留空
  catalognumber={分类号}, % 可以留空
  %
  %=========
  % 英文信息
  %=========
  etitle={Research and Application on Industrial Time Series Data Feature Learning},
  % 这块比较复杂,需要分情况讨论:
  % 1. 学术型硕士
  %    edegree:必须为Master of Arts或Master of Science(注意大小写)
  %             “哲学、文学、历史学、法学、教育学、艺术学门类,公共管理学科
  %              填写Master of Arts,其它填写Master of Science”
  %    emajor:“获得一级学科授权的学科填写一级学科名称,其它填写二级学科名称”
  % 2. 专业型硕士
  %    edegree:“填写专业学位英文名称全称”
  %    emajor:“工程硕士填写工程领域,其它专业学位不填写此项”
  % 3. 学术型博士
  %    edegree:Doctor of Philosophy(注意大小写)
  %    emajor:“获得一级学科授权的学科填写一级学科名称,其它填写二级学科名称”
  % 4. 专业型博士
  %    edegree:“填写专业学位英文名称全称”
  %    emajor:不填写此项
  edegree={Master of Engineering},
  emajor={Software Engineering},
  eauthor={Huang Fanling},
  esupervisor={Associate Research Fellow Deng Yangdong},
  %eassosupervisor={Chen Wenguang},
  % 日期自动生成,若需指定按如下方式修改:
  % edate={December, 2005}
  %
  % 关键词用“英文逗号”分割
  ckeywords={时间序列特征学习, 故障预测与健康管理技术, 符号回归, 复杂系统临界相变, 生成式对抗网络},
  ekeywords={Time Series Feature Learning, Prognostics and Health Management, Symbolic Regression, Critical Transitions of Complex System, Generative Adversarial Networks}
}

% 定义中英文摘要和关键字
\begin{cabstract}
  
   工业系统是国民经济和国家安全的重要基础,普遍集成度高、耦合关系复杂,任何零部件出现故障都可能导致重大损失。维修技术是保障工业系统可靠、安全、可持续运行的重要手段。在全球化趋势下获得清晰准确的系统机理知识困难甚至不可能,本文认为具备知识发现能力的数据驱动的故障预测和健康管理(Prognostics and Health Management,PHM)技术是工业系统全面实现视情维修的关键出路。时间序列数据是工业系统领域的重要资源,然而因存在高维、高噪、时变等特点使其难以被开采利用。如何学习时间序列数据的有效特征一直是研究热点和难点。
  % 唯一出路?重要出路?

  %本文在《列车多物理域信号分析与健康状态检测仪器》自然科学重大仪器项目支持下
  %本文基于时间序列的可列可加性将 时间序列分成操作趋势和随机波动信号两部分。
  %本文从实际工程出发以数据驱动的方法为主要手段探索工业时间序列特征学习新方法。
  本文从实际工程出发探索工业时间序列特征学习新方法。
  (1)针对实际采集的多维车轴温度数据,本文提出了基于符号回归的系统动态特征学习框架,并融合遗传算法和确定性优化算法训练模型。实验表明此框架学习的系统方程有效反应了系统的动态特性和各信号间的相互作用关系。进一步,基于系统方程构建时间序列预测模型形成了完备的在线实时异常检测系统。此工作证明了现代数据挖掘技术可以通过温度数据揭示轴承的动态性,并可基于此执行精确的预测。
  (2)针对少量系统失效敏感时间序列数据,本文首次引入复杂系统临界相变预测理论,提出基于复杂系统临界相变理论的系统失效早期预警特征学习框架对时间序列的随机波动信号进行利用,在4个不同类型的工程系统数据集上发现了一致的系统失效早期预警特征;进一步通过多种方法验证了特征的有效性以及系统失效和临界相变的一致性;此发现为研究通用方法来预测和预防由工程系统失效引起的潜在灾难开辟了新道路。
  (3)本文首次基于生成式对抗网络(Generative Adversarial Nets,GAN)提出了针对时间序列特征学习的通用模型时间序列生成式对抗网络(Time Series GAN,TSGAN),并在最大的时间序列分类数据集(85个时间序列数据集)上验证模型的有效性。一方面,TSGAN的生成器可有效学习时间序列数据集的分布,并生成逼真的多样化的时间序列;另一方面,TSGAN的判别器作为特征转换器与简单分类器结合形成半监督式时间序列分类框架并取得优秀的分类效果。此工作为无监督式时间序列特征学习以及基于此的半监督式机器学习提供了新思路。 

  本文的所有特征均得到了有效应用,对于深入理解复杂工业系统运行机理,发现并预测潜在故障隐患,建立依靠先进技术支撑的维修体制,具有重大意义。


  %本文的研究对于深入理解复杂工业系统运行机理,发现并预测潜在故障隐患,建立依靠先进技术支撑的维修体制,具有重大意义。

% 进而实现保障系统安全、提高维修效率、减少维修成本、降低系统故障带来的损失等目标
%   本文的创新点主要有:
%   \begin{itemize}
%     \item 基于符号回归的动态特征学习与应用呢;
%     \item 基于复杂系统相变理论的系统失效早期预警特征学习与应用;
%     \item 基于生成式对抗网络的时间序列通用特征学习与应用。
%   \end{itemize}
\end{cabstract}

% 如果习惯关键字跟在摘要文字后面,可以用直接命令来设置,如下:
% \ckeywords{\TeX, \LaTeX, CJK, 模板, 论文}

\begin{eabstract}
Industrial systems are the crucial foundation of national economy and security. 
Most of the industrial systems are highly integrated and complexly coupled, and failure of any part of them can cause huge losses.  
 Maintenance techniques play an important role in ensuring the industrial systems operate reliably, safely and sustainably. 
% Maintenance technology play an important role in ensuring the reliable, safe and sustainable operation of industrial system.
However, under the trend of globalization, it’s pretty difficult to obtain clear and accurate operational knowledge of the industrial systems. 
%The globalization of the manufacturing process and the complex equipment ownership relations complicates the process to establish a comprehensive knowledge base of the domain knowledge.
% To address such challenges, this thesis asserts that the data-driven PHM technique with the ability of knowledge discovery is the key to achieve on-condition maintenance. 
To address such challenges, this thesis asserts that the data-driven Prognostics and Health Management (PHM) technique covering both knowledge discovery and prognostics is the only viable path toward developing an overall condition-based maintenance for industrial system.
Time-series data, the significant resource in industrial systems, is hard to be utilized because of its characteristics of high dimension, heavy noise and time variance.  It is always a challenging but active research subject to learn valuable features of time-series data.

In this paper, several new methods of industrial time-series feature learning were explored from the perspective practical engineering. 
(1) For the multi-dimensional axle temperature data, a framework based on symbolic regression was proposed for system dynamic features learning. The model was trained with genetic algorithm and deterministic optimization algorithm. The experimental results suggest that the dynamic equations learned from framework can effectively reflect not only the dynamic characteristics of the system but also the interaction among the signals. Furthermore, a comprehensive online real-time anomaly detection system was formed, by constructing a time-series prediction model based on the dynamic equation. This work proves that data mining techniques can expose the dynamics of axles through temperature data, and then perform accurate prediction.
(2) Aiming at small amounts of system failure sensitive time-series data, the critical transition prediction theory of complex system was first introduced in the thesis, a fault early warning feature learning framework was proposed to utilize the random fluctuation signals of time-series. And the consistent early warning features was found on four different engineering system data sets; Furthermore, the validity of the features and the consistency of the system failure and critical transition were verified by different methods. Such a discovery paves the way toward a generic methodology to predict and prevent potential disasters caused by the failure of engineered systems. 
(3) A generic unsupervised time-series feature learning model based on Generative Adversarial Nets (GAN) was proposed in this thesis for the first time, called Time-Series Generative Adversarial Nets (TSGAN). The validity of the TSGAN was verified on the largest time-series classification datasets (85 time-series datasets). On the one hand, the generator of TSGAN can effectively learn the distribution of time-series datasets and generate realistic and diversified time-series. On the other hand, the discriminator of TSGAN as a feature encoder can be combined with a simple classifier to form a semi-supervised time-series classification framework, which gains excellent result. This method of work provides a new idea for unsupervised time-series feature learning and semi-supervised machine learning based on it.

All above time-series features have been effectively applied, which is of great significance to further understanding the operation mechanism of complex industrial systems, diagnosing and prognosing potential failure and establishing maintenance system supported by advanced technology.
\end{eabstract}

% \ekeywords{\TeX, \LaTeX, CJK, template, thesis}
